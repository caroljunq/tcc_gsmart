% ------------------------------------------------------------------------
% ------------------------------------------------------------------------
% Monografia 2017
% Trabalho de Conclusão de Curso
% Baseia-se no documento modelo de TCC do abntex2
% Para saber mais, acesse https://github.com/abntex/abntex2
% ------------------------------------------------------------------------
% ------------------------------------------------------------------------

\documentclass[
		% -- opções da classe memoir --
		12pt,				% tamanho da fonte
		openright,			% capítulos começam em pág ímpar (insere página vazia caso preciso)
		oneside,			% para impressão em verso e anverso. Oposto a oneside
		a4paper,			% tamanho do papel.
		% -- opções da classe abntex2 --
		chapter=TITLE,		% títulos de capítulos convertidos em letras maiúsculas
		%section=TITLE,		% títulos de seções convertidos em letras maiúsculas
		%subsection=TITLE,	% títulos de subseções convertidos em letras maiúsculas
		%subsubsection=TITLE,% títulos de subsubseções convertidos em letras maiúsculas
		% -- opções do pacote babel --
		english,			% idioma adicional para hifenização
		brazil				% o último idioma é o principal do documento
	]{abntex2}


% ----------------------------------------------------------
% Pacotes básicos
% ----------------------------------------------------------
%\usepackage{helvet}
\usepackage[scaled]{helvet}
\renewcommand*\familydefault{\sfdefault} 	% Only if the base font of the document is to be sans serif
										% Foi necessário para acertar o documento, continha diversos erros
\usepackage[T1]{fontenc}		% Selecao de codigos de fonte.
\usepackage[utf8]{inputenc}	% Codificacao do documento (conversão automática dos acentos)
\usepackage{lastpage}		% Usado pela Ficha catalográfica
\usepackage{indentfirst}		% Indenta o primeiro parágrafo de cada seção.
\usepackage{color}			% Controle das cores
\usepackage{graphicx}		% Inclusão de gráficos
\usepackage{microtype} 		% para melhorias de justificação
% ----------------------------------------------------------


% ----------------------------------------------------------
% Pacotes adicionais, usados apenas no âmbito do Modelo Canônico do abnteX2
%% ----------------------------------------------------------
\usepackage{lipsum}				% para geração de dummy text
\usepackage{customizacoes} 		% customizações feitas pelo autor
% ----------------------------------------------------------
\usepackage{pdfpages}

% ----------------------------------------------------------
% Pacotes de citações
% ----------------------------------------------------------
\usepackage[alf]{abntex2cite}				% Citações padrão ABNT

% ----------------------------------------------------------
% CONFIGURAÇÕES DE PACOTES
% ----------------------------------------------------------

% ----------------------------------------------------------
% Configurações do pacote backref
% ----------------------------------------------------------
\definecolor{thered}{rgb}{0.65,0.04,0.07}
\definecolor{thegreen}{rgb}{0.06,0.44,0.08}
\definecolor{thegrey}{gray}{0.5}
\definecolor{theshade}{rgb}{1,1,0.97}
\definecolor{theframe}{gray}{0.6}
% ----------------------------------------------------------



% ----------------------------------------------------------
% Informações de dados para CAPA e FOLHA DE ROSTO
% ----------------------------------------------------------

\titulo{GSMART - Uma ferramenta de geomarketing para a medição do tráfego de pessoas através de redes Wi-Fi}
\autor{Carolina Junqueira Ferreira \\ Juliana D'alessio Grandini}
\local{Bauru}
\data{2017}
\orientador{Prof. Dr. Sidnei Bergamaschi}

\instituicao{%
  Universidade Estadual Paulista "Júlio de Mesquita Filho"
  \par
  Faculdade de Ciências - Campus Bauru
  \par
  Departamento de Computação
}
\tipotrabalho{Monografia (Trabalho de Conclusão de Curso)}
% O preambulo deve conter o tipo do trabalho, o objetivo,
% o nome da instituição e a área de concentração
% foi necessário utilizar \~{a} e etc para os acentos por problemas na geração do PDF
\preambulo{Trabalho de Conclus\~{a}o do Curso de Bacharelado em Sistemas de Informação apresentado ao Departamento de Computa\c{c}\~{a}o da Faculdade de Ci\^{e}ncias da Universidade Estadual Paulista ``J\'{ú}lio de Mesquita Filho'' – UNESP, C\^{a}mpus de Bauru.}

% ----------------------------------------------------------


% ----------------------------------------------------------
% Configurações de aparência do PDF final
% ----------------------------------------------------------

% alterando o aspecto da cor azul
\definecolor{blue}{RGB}{0,0,0}

% informações do PDF
\makeatletter
\hypersetup{
     	%pagebackref=true,
		pdftitle={\@title},
		pdfauthor={\@author},
    	pdfsubject={\imprimirpreambulo},
	    pdfcreator={LaTeX with abnTeX2},
		pdfkeywords={geomarketing}{redes Wi-Fi}{trafego de pessoas}{Redes Wi-Fi}{Dispositivos Moveis}{abntex2}{trabalho acadêmico},
		colorlinks=true,       		% false: boxed links; true: colored links
    	linkcolor=blue,          	% color of internal links
    	citecolor=blue,        		% color of links to bibliography
    	filecolor=magenta,      		% color of file links
		urlcolor=blue,
		bookmarksdepth=4
}
\makeatother
% ----------------------------------------------------------


% ----------------------------------------------------------
% Espaçamentos entre linhas e parágrafos
% ----------------------------------------------------------

% O tamanho do parágrafo é dado por:
\setlength{\parindent}{1.3cm}

% Controle do espaçamento entre um parágrafo e outro:
\setlength{\parskip}{0.2cm}  % tente também \onelineskip

% ----------------------------------------------------------
% compila o indice
% ----------------------------------------------------------
\makeindex
% ----------------------------------------------------------


% ----------------------------------------------------------
% Configurações de projeto
% ----------------------------------------------------------
\newif\iffinal
\finaltrue % define se é um arquivo final, se for não for retira umas partes.

\newif\ifrelatorio
\relatoriofalse % define se é um arquivo final, se for não for retira umas partes.

\newif\ifabstract
\abstractfalse % define se mostra o abstract em inglês ou não.

\newif\ifresumo
\resumotrue % define se mostra o resumo ou não.

\newif\ifficha
\fichafalse % define se mostra a ficha catalográfica ou não
% ----------------------------------------------------------


% ----------------------------------------------------------
% Início do documento
% ----------------------------------------------------------
\begin{document}

% Seleciona o idioma do documento (conforme pacotes do babel)
%\selectlanguage{english}
\selectlanguage{brazil}

% Retira espaço extra obsoleto entre as frases.
\frenchspacing

% ----------------------------------------------------------
% ELEMENTOS PRÉ-TEXTUAIS
% ----------------------------------------------------------
\pretextual


% ----------------------------------------------------------
% Capa
% ----------------------------------------------------------
\imprimircapa
% ----------------------------------------------------------


% ----------------------------------------------------------
% Folha de rosto
% (o * indica que haverá a ficha bibliográfica)
% ----------------------------------------------------------
\imprimirfolhaderosto
% ----------------------------------------------------------


% ----------------------------------------------------------
% Inserir a ficha catalográfica
% ----------------------------------------------------------

% Isto é um exemplo de Ficha Catalográfica, ou "Dados internacionais de
% catalogação-na-publicação''. Você pode utilizar este modelo como referência.
% Porém, provavelmente a biblioteca da sua universidade lhe fornecerá um PDF
% com a ficha catalográfica definitiva após a defesa do trabalho. Quando estiver
% com o documento, salve-o como PDF no diretório do seu projeto e substitua todo
% o conteúdo de implementação deste arquivo pelo comando abaixo:
%
% \begin{fichacatalografica}
%     \includepdf{fig_ficha_catalografica.pdf}
% \end{fichacatalografica}

%\begin{fichacatalografica}
%	\sffamily
%	\vspace*{\fill}					% Posição vertical
%	\begin{center}					% Minipage Centralizado
%		\fbox{\begin{minipage}[c][8cm]{13.5cm}		% Largura
%				\small
%				\imprimirautor
%				%Sobrenome, Nome do autor
%
%				\hspace{0.5cm} \imprimirtitulo / \imprimirautor. --
%				\imprimirlocal, \imprimirdata-
%
%				\hspace{0.5cm} \pageref{LastPage} p. : il. (algumas color.) ; 30 cm.\\
%
%				\hspace{0.5cm} \imprimirorientadorRotulo~\imprimirorientador\\
%
%				\hspace{0.5cm}
%				\parbox[t]{\textwidth}{\imprimirtipotrabalho~--~\\ \imprimirinstituicao,
%					\imprimirdata.}\\
%
%				\hspace{0.5cm}
%				1. Balanceamento de Cargas.
%				2. Redes Neurais Artificiais.
%				3. Computação em Nuvem.
%				I. \imprimirorientador.
%				II. Universidade Estadual Paulista "Júlio de Mesquita Filho".
%				III. Faculdade de Ciências.
%				IV. \imprimirtitulo
%			\end{minipage}}
%		\end{center}
%	\end{fichacatalografica}

%\begin{fichacatalografica}
%	\ttfamily
%	\vspace*{\fill}					% Posição vertical
%	\hspace{1.5cm}
%	\begin{centering}	% Minipage Centralizado
%		\fbox{\begin{minipage}[c][7.5cm]{12.5cm}		% Largura
%				\footnotesize
%				\vspace{0.3cm}
%				\hspace{2.0cm} Setoue, Karoline Kimiko Figueiredo.
%				%Sobrenome, Nome do autor
%
%				\hspace{2.0cm} \parbox[t]{\textwidth}{\hspace{0.5cm} Aplicação de redes neurais artificiais no balanceamento de carga em serviços de nuvem / \imprimirautor, \imprimirdata} \\
%
%				\hspace{2.0cm} \parbox[t]{\textwidth}{\hspace{0.5cm} \pageref{LastPage} f. : il.} \\
%
%				\hspace{2.0cm} \parbox[t]{\textwidth}{\hspace{0.5cm} \imprimirorientador} \\
%
%				\hspace{2.0cm} \parbox[t]{\textwidth}{\hspace{0.5cm} Monografia (Graduação)~--~Universidade Estadual \\ Paulista. Faculdade de Ciências, Bauru, 2017 \\}
%				\\
%
%
%				\hspace{2.0cm} \parbox[t]{\textwidth}{\hspace{0.5cm} 1. Balanceamento de carga. 2. Redes neurais artificiais. \\ 3. Computação em nuvem. I. Universidade \\ Estadual Paulista. Faculdade de Ciências. II. Título.}
%			\end{minipage}}
%		\end{centering}
%\end{fichacatalografica}

 %\begin{fichacatalografica}
    % \includepdf{ficha.pdf}
 %\end{fichacatalografica}
% ----------------------------------------------------------


% ----------------------------------------------------------
% Inserir errata
% ----------------------------------------------------------
%\begin{errata}
%Elemento opcional da \citeonline[4.2.1.2]{NBR14724:2011}. Exemplo:

%\vspace{\onelineskip}

%FERRIGNO, C. R. A. \textbf{Tratamento de neoplasias ósseas apendiculares com
%reimplantação de enxerto ósseo autólogo autoclavado associado ao plasma
%rico em plaquetas}: estudo crítico na cirurgia de preservação de membro em
%cães. 2011. 128 f. Tese (Livre-Docência) - Faculdade de Medicina Veterinária e
%Zootecnia, Universidade de São Paulo, São Paulo, 2011.

%\begin{table}[htb]
%\center
%\footnotesize
%\begin{tabular}{|p{1.4cm}|p{1cm}|p{3cm}|p{3cm}|}
%  \hline
%   \textbf{Folha} & \textbf{Linha}  & \textbf{Onde se lê}  & \textbf{Leia-se}  \\
%    \hline
%    1 & 10 & auto-conclavo & autoconclavo\\
%   \hline
%\end{tabular}
%\end{table}

%\end{errata}
% ----------------------------------------------------------


% ----------------------------------------------------------
% Inserir folha de aprovação
% ----------------------------------------------------------

% Isto é um exemplo de Folha de aprovação, elemento obrigatório da NBR
% 14724/2011 (seção 4.2.1.3). Você pode utilizar este modelo até a aprovação
% do trabalho. Após isso, substitua todo o conteúdo deste arquivo por uma
% imagem da página assinada pela banca com o comando abaixo:
%
% \includepdf{folhadeaprovacao_final.pdf}
%
%\begin{folhadeaprovacao}

	%\begin{center}
		%{\ABNTEXchapterfont\large\imprimirautor}

		%\vspace*{\fill}\vspace*{\fill}
		%\begin{center}
			%\ABNTEXchapterfont\bfseries\Large\imprimirtitulo
		%\end{center}
		%\vspace*{\fill}

		%\hspace{.45\textwidth}
		%\begin{minipage}{.5\textwidth}
			%\imprimirpreambulo
		%\end{minipage}%
		%\vspace*{\fill}
	%%\end{center}

	%\center Banca Examinadora
	%\begin{center}
		%\vspace*{0.5cm}
		%\textbf{\imprimirorientador} \\ Orientador \\ Universidade Estadual Paulista "Júlio de Mesquita Filho" \\ Departamento de computação\\ Faculdade de Ciências \\
	%\end{center}
	%\begin{center}
		%\vspace*{0.5cm}
		%\textbf{Profa. Dra. Simone das Graças Domingues Prado} \\ Universidade Estadual Paulista "Júlio de Mesquita Filho" \\ Departamento de computação\\ Faculdade de Ciências \\
	%\end{center}
	%\begin{center}
		%\vspace*{0.5cm}
		%\textbf{Profa. Dra. Roberta Spolon} \\ Universidade Estadual Paulista "Júlio de Mesquita Filho" \\ Departamento de computação\\ Faculdade de Ciências \\
	%\end{center}
	%\assinatura{\textbf{Professor} \\ Convidado 3}
	%\assinatura{\textbf{Professor} \\ Convidado 4}

	%\begin{center}
		%\vspace*{0.5cm}
		%\par
		%{Bauru, 07 de Fevereiro de 2017.}
		%\vspace*{1cm}
	%\end{center}

%\end{folhadeaprovacao}
% ----------------------------------------------------------


% ----------------------------------------------------------
% Dedicatória
% ----------------------------------------------------------
%\ifrelatorio
	%\begin{dedicatoria}
		%\vspace*{\fill}
		%\centering
		%\noindent
		%\textit{ Este trabalho é dedicado às crianças adultas que,\\
		%		quando pequenas, sonharam em se tornar cientistas.}
		%\vspace*{\fill}
	%\end{dedicatoria}
%\fi
% ----------------------------------------------------------


% ----------------------------------------------------------
% Agradecimentos
% ----------------------------------------------------------
%\iffinal
	%\begin{agradecimentos}

	%\end{agradecimentos}
%\fi
% ----------------------------------------------------------


% ----------------------------------------------------------
% Epígrafe
% ----------------------------------------------------------
%\iffinal
	%\begin{epigrafe}
		%\vspace*{\fill}
		%	\begin{flushright}


		%\textit{"Mundo mundo vasto mundo..."\\
		%	(Carlos Drummond de Andrade)}
		%%\textit{""\\
		%		}
		%\end{flushright}
	%\end{epigrafe}
%\fi
% ----------------------------------------------------------


% ----------------------------------------------------------
% RESUMOS
% ----------------------------------------------------------
\ifresumo
	% resumo em português
	\setlength{\absparsep}{18pt} % ajusta o espaçamento dos parágrafos do resumo
	\begin{resumo}
			Baseando-se em conceitos de \emph{geomarketing}, este trabalho tem como objetivo
			desenvolver um sistema que mede o tráfego de pessoas em determinadas zonas através da rede Wi-Fi e dispositivos móveis.

		\textbf{Palavras-chave}: \emph{Geomarketing}. Tráfego de pessoas. Redes Wi-Fi. Dispositivos Móveis.

	\end{resumo}

	% resumo em inglês
	\begin{resumo}[Abstract]
	 	\begin{otherlanguage*}{english}
    Based on the concepts of geomarketing, this work aims to develop a
    system that measures the traffic of people in certain zones using
    Wi-Fi network and mobile devices.

		\textbf{Keywords}: Geomarketing. People Traffic. Wi-Fi Networks. Mobile Devices.

		\end{otherlanguage*}

	\end{resumo}

\fi
% ----------------------------------------------------------


% ----------------------------------------------------------
% inserir lista de ilustrações
% ----------------------------------------------------------
\iffinal
	\pdfbookmark[0]{\listfigurename}{lof}
	\listoffigures*
	\cleardoublepage
\fi
% ----------------------------------------------------------


% ----------------------------------------------------------
% inserir lista de tabelas
% ----------------------------------------------------------
\iffinal
	\pdfbookmark[0]{\listtablename}{lot}
	\listoftables*
	\cleardoublepage
\fi
% ----------------------------------------------------------


% ----------------------------------------------------------
% inserir lista de abreviaturas e siglas
% ----------------------------------------------------------
\iffinal
	\begin{siglas}
			\item[AP] Access Point - Ponto de Acesso
      \item[CSS] Cascading Style Sheets
      \item[CSV] Comma-Separated Values - Valores separados por vírgula
      \item[HTML] HyperText Markup Language - Linguagem de Marcação de Hipertexto
      \item[HTTP] Hypertext Transfer Protocol - Protocolo de Transferência de Hipertexto
			\item[IoT] Internet of Things - Internet das Coisas
      \item[JSON] JavaScript Object Notation - Notação de Objetos JavaScript
			\item[MAC] Media Access Control 
			\item[NIC] Network Interface Card - Placa de Interface de Rede
			\item[RFID] Radio-Frequency IDentification - Identificação por radiofrequência
			\item[ROO] Return on Objectives - Retorno dos Objetivos (Métrica)
			\item[TI] Tecnologia da Informação
			\item[Wi-Fi] Marca registrada da Wi-Fi Alliance. Rede local sem fios baseados no padrão IEEE 802.11


	\end{siglas}
\fi
% ----------------------------------------------------------


% ----------------------------------------------------------
% inserir lista de símbolos
% ----------------------------------------------------------
%\iffinal
%	\begin{simbolos}
%			\item[$ \Gamma $] Letra grega Gama
%			\item[$ \Lambda $] Lambda
%			\item[$ \zeta $] Letra grega minúscula zeta
%			\item[$ \in $] Pertence
%	\end{simbolos}
%\fi
% ----------------------------------------------------------


% ----------------------------------------------------------
% inserir o sumario
% ----------------------------------------------------------
\pdfbookmark[0]{\contentsname}{toc}
\tableofcontents*
\cleardoublepage
% ----------------------------------------------------------



% ----------------------------------------------------------------------------------------------------------------------------------



% ----------------------------------------------------------------------------------------------------------------------------------
% ELEMENTOS TEXTUAIS
% ----------------------------------------------------------------------------------------------------------------------------------
\textual



\chapter{Introdução}
\label{introducao}

Teste

\section{Justificativa}
\label{justificativa}

Lorem ipsum

\section{Objetivos}
\label{objetivos}

lorem ipsum
 % inclui o arquivo introducao.tex


\chapter{Fundamentação Teórica}
\label{fundamentacao-teorica}

\section {Geomarketing}
O Geomarketing pode ser entendido como uma ferramenta de análise estatística de dados, com intuito de localizar padrões que possam ser utilizados e combinados na elaboracao de indicadores, perfis de consumo e estrategias de negócios, de modo a auxiliar na tomada de decisões. Geralmente, o servico é oferecido por consultorias especializadas - o  objetivo da empresa que contratante é a melhoria no desempenho de seu negócio.

O Geomarketing é ainda pouco difundido no Brasil, no entanto cada vez mais se populariza no âmbito dos negócios: segundo \cite{Exame}, utilizado de forma amadora há 20 anos, o uso de ferramentas de localização geográfica evoluiu e alcançou importância dentro da estratégia de expansão das empresas: grupos como Coca-Cola e O Boticário usam o marketing geográfico e pequenas e médias empresas já começam a mirar em sistemas de busca com foco na geolocalização.  % inclui o arquivo fundamentacao.tex


\chapter{Metodologia}
\label{metodologia}

Teste
 % inclui o arquivo metodologia.tex

\chapter{Cronograma}
\label{cronograma}
O cronograma do trabalho está divido em duas tabelas: \autoref{tcc-1} sobre as entregas na disciplina de TCCI
e a \autoref{tcc-2} as entregas na disciplina de TCCII.

\begin{table}[h]
  \begin{center}
  	\caption{\label{tcc-1}Cronograma de atividades para o TCCI}
  	\begin{tabular}{c{4cm} c{2cm} c{2cm} c{2cm} c{2cm} c{2cm}}
      \hline
      Atividade & Abr & Mai & Jun & Jul & Ago \\
      \hline
      \hline
      Levantamento bibliográfico inicial & X & X & X \\
      \hline
      Definição das tecnologias e dos provedores & X & X & X \\
      \hline
      Documentação da arquitetura do sistema & & X & X \\
      \hline
      Desenvolver módulo sensor & & & X & X & X \\
      \hline
      Aplicar pré-processamento dos dados & & & & X & X \\
      \hline
      Teste e documentação do que foi desenvolvido & & & & X & X \\
      \hline
  	\end{tabular}
  	\legend{Fonte: Elaborado pelas autoras.}
  \end{center}
\end{table}

\begin{table}[h]
  \begin{center}
  	\caption{\label{tcc-2}Cronograma de atividades para o TCCII}
  	\begin{tabular}{c{5cm} c{2cm} c{2cm} c{2cm} c{2cm} c{2cm}}
      \hline
      Atividade & Set & Out & Nov & Dez & Jan \\
      \hline
      \hline
      Preparação do servidor & X & X \\
      \hline
       Envio de dados para o servidor &  & X & X \\
      \hline
      Teste de validação do servidor e sensor & & X & X & X \\
      \hline
      Desenvolver interface & & & & X & X  \\
      \hline
      Integrar componentes da arquitetura & & & X & X & X \\
      \hline
      Teste finais e validação & & & & & X \\
      \hline
  	\end{tabular}
  	\legend{Fonte: Elaborado pelas autoras.}
  \end{center}
\end{table}


\chapter{Conclusão}
\label{conclusao}
 % inclui o arquivo conclusao.tex

% ----------------------------------------------------------
% ELEMENTOS PÓS-TEXTUAIS
% ----------------------------------------------------------

% ----------------------------------------------------------
% Referências bibliográficas
% ----------------------------------------------------------
\bibliography{referencias}
% ----------------------------------------------------------

\end{document}

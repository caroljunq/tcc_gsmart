
\chapter{Materiais e Métodos}
\label{metodologia}
Este capítulo descreve o modo como a construção do sistema proposto na \autoref{objetivos} será conduzida.

\section{Tecnologias utilizadas}
A escolhas das tecnologias seguem de acordo com a arquitetura do sistema.

\subsection{Sensor}
Para a detecção de dispositivos móveis será utilizado um Raspberry Pi Model 3 B juntamente com um adaptador Wi-Fi. O Raspberry foi escolhido,
pois oferece interface amigável de programação (Debian Jessie); possui poder de processamento para receber os milhares de pacotes, pré-processá-los
e enviar para a \emph{cloud}; possui entradas USB que podem receber um modem 3G e entrada Ethernet e seu tamanho pequeno \cite{rpi2017}. Apesar de possuir
antena Wi-Fi, o adaptador será usado para ampliar a área de cobertura do sinal. Já foram adquiridos dois aparelhos e dois adaptadores, então
nenhum custo de hardware será necessário.

Outras opções foram consideradas por serem baratas, acessíveis e terem documentação aberta. Foi o caso do ESP8266 que é possui um tamanho extremamente
reduzido e possui o custo médio de R\$15,00 \cite{Embarcados2015}, mas seu uso para este trabalho fica impossibilitado. Isso
ocorre, pois essa tecnologia não consegue ser habilitada para o modo monitor da interface de rede que é essencial para detectar dispositivos móveis
\cite{Puhl2016} \cite{Ferreira2016}.

\subsection{Comunicação Wi-Fi}
A comunicação entre dispositivos móveis e o sensor será feita através do protocolo de rede Tshark \cite{Wireshark2017}. Este é uma versão de terminal do protocolo
analisador de rede Wireshark \cite{Wireshark2017a}. Esses protocolos permitem realizar o estudo da rede a partir do recebimento de pacotes e seus campos. Por
ser uma tecnologia aberta e iniciada em 1998, possui ampla documentação e maturidade, tanto que detectar aparelhos na rede através dessa comunicação
está presente em trabalhos como o de \citeonline{Puhl2016} e de \citeonline{Ferreira2016}.

\subsection{Objeto identificador}
O objeto que distinguirá as pessoas para a contagem será seu dispositivo móvel. Os motivos pela escolha são apresentado na \autoref{dispositivo-coisa}.

\subsection{Processamento}
Com o poder de processamento do Raspberry Pi, um pré-processamento dos dados captados será feito. Estes então serão enviados ao servidor onde
ocorrerá o agrupamento para fins estatísticos. O servidor ainda precisa ser adquirido.

\subsection{Interface Web}
Para apresentar os dados estatísticos sobre o tráfego de pessoas e seus perfis, uma interface Web será desenvolvida com as tecnologias
HTML5, Material Design e Javascript.

\section{Pontos de Entrega}
\label{pontos-entrega}
Os módulos assemelham-se à arquitetura do sistema proposto. Os módulos a serem desenvolvidos são:
\begin{itemize}
  \item \textbf{Módulo Sensor:} a detecção de aparelhos móveis pelo Raspberry Pi;
  \item \textbf{Módulo Pré-processamento:} os dados captados pelo sensor estão prontos para serem enviados ao servidor;
  \item \textbf{Módulo Servidor:} preparação do servidor para receber os dados do sensor;
  \item \textbf{Módulos Integração:} integração da comunicação do Raspberry Pi e do Servidor, ou seja, transmissão de dados;
  \item \textbf{Módulo Interface:} conectar servidor e interface apresentando os dados recolhidos.
\end{itemize}

\section{Testes e validação do projeto}
Para a validação do sistema proposto serão realizados testes de "caixa-branca" e "caixa-preta". Os testes de caixa-preta incluem testar:
\begin{itemize}
  \item a comunicação entre sensor e os dispositivos móveis;
  \item comunicação entre Raspberry Pi e servidor;
  \item se os dados capturados são os desejados;
  \item se o processamento dos dados capturados estão corretos.
  \item comunicação servidor e interface.
\end{itemize}

Já os testes de caixa-branca são resumidos na usabilidade do sistema. Ambiente controlados e não controlados serão selecionados
para testar a precisão na determinação do tráfego de pessoas e dos perfis, então validar se o sistema proposto funciona ou não.
O sistema final vai ser considerado aplicável ou não como técnica de geomarketing caso uma margem de erro a ser determinada fique dentro
dos limites. Os ambientes controlados correspondem a locais em que sei o número de pessoas passando a cada instante e qual o modelo de celular
de cada uma. Já os ambientes não controlados são locais onde o tráfego e contagem será feita de maneira manual e comparada a do sistema.

Inicialmente, visa-se desenvolver os primeiros testes em ambiente
controlado, numa área pequena e com poucos dispositivos móveis, para verificar o
comportamento do sistema desenvolvido na medição do tráfego. Após testes
iniciais, pretende-se encontrar uma organização parceira que esteja dentro das
especificações necessárias e deseje conhecer melhor seu público alvo, cedendo
seu espaço e sua rede para alguns procedimentos e testes com a aplicação proposta
 - nessa etapa, o projeto busca verificar o desempenho do
sistema em ambiente real, com maior quantidade de dispositivos móveis e de
fabricantes.

\section{Análise de Riscos}
Considerando as premissas dos tópicos anteriores, há alguns itens e áreas que podem sofrer desvios ao longo do trabalho. Estes itens e seus planos
de contingência respectivamente são:

\begin{itemize}
  \item Falha da detecção de dispositivos (precisão): serão feitas duas formas de detecção através do protocolo Tshark, as duas garantem que os dados
  de uma e outra são verídicos, caso uma falhe há a outra para detectar os aparelhos móveis;
  \item Processamento no servidor é complexa: caso o desenvolvimento do processamento de dados no servidor seja muito complexa e considerando
  que trabalharemos com estatísticas, optar por uma \emph{cloud} seria uma opção;
  \item Raspberry Pi perder a conexão com a rede para mandar dados ao servidor: um backup dos dados será feito no aparelho, e então quando
  a conexão retornar, esses serão enviado ao servidor. Também há a possibilidade do uso de um Modem 3G para garantir o envio.
\end{itemize}

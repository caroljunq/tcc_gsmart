
\chapter{Materiais e Métodos}
\label{metodologia}
Essa seção descreverá brevemente as tecnologias e metodologias empregadas na realização deste trabalho.

\section{Tecnologias utilizadas}
\label{tecnologias-usadas}

\subsection{Comunicação Wi-Fi e \emph{probe request}}
Segundo \cite{Teleco2008}, uma Wireless LAN (WLAN) é uma rede local sem fio padronizada pelo IEEE 802.11. É conhecida também pelo nome de Wi-Fi, abreviatura de ‘wireless fidelity’ (fidelidade sem fios) e marca registrada pertencente à Wireless Ethernet Compatibility Alliance (WECA). 

De acordo com \cite{SIMOES2015}, A tecnologia Wi-Fi é muito utilizada em sistemas de posicionamento indoor, pois não existe a necessidade de criar uma infraestrutura de comunicação: em praticamente todos os espaços fechados com afluência de pessoas, existe já uma criada. Vale ressaltar, porém, que redes Wi-Fi estão sujeitas a pontos cegos (desvios de sinal em áreas não atingidas pela rede) - isso pode ser causados por interferências de outros equipamentos, objetos, fiação elétrica ou mesmo paredes. Dispositivos a uma mesma distância de um ponto de acesso podem receber qualidades de sinal diferentes. 

Ainda segundo \cite{SIMOES2015}, um campo de informação que pode ser recolhido pelos dispositivos móveis é o MAC Address, que permite identificar, na rede, o AP ao qual o aparelho está ligado, característica esta utilizada no desenvolvimento deste trabalho. 

Como explanado por \cite{Teleco2016}, para permitir que uma estação móvel comunique-se com outras em uma rede IBSS (Independent Basic Service Set) ou um AP em uma rede infra-estrutura BSS (Basic Service Set), ela deve primeiramente encontrá-las: esse processo é conhecido como \emph{varredura} e pode ser de 2 tipos: Passivo, modalidade envolvendo somente a “escuta” de tráfego, e Ativo (método utilizado neste trabalho), no qual a estação X executa uma varredura para extrair informações das demais estações e dos AP’s, economizando tempo. Para tanto, a estação ativamente transmite \emph{queries} (perguntas para extrair as respostas das estações numa BSS), movendo-se então para um canal e transmitindo um quadro do tipo \emph{probe request} (requisição de sondagem): se houver BSS no canal que coincida com o SSID (Service Set Identifier) do quadro “requisição de sondagem”, a estação irá responder, enviando um quadro \emph{probe response} (resposta de sondagem) para a estação que fez a pergunta.

\subsection{Modo monitor}
\label{modo-monitor}

Para que um dispositivo móvel seja identificado independente de qualquer rede ou AP, é essencial que o sistema de detecção possua uma placa e/ou adaptador de rede (NIC) que possa ser habilitado para o modo monitor.

Geralmente, uma interface de rede qualquer captura pacotes dos tipos \emph{managed} e \emph{beacons} que são originados por APs. Estes pacotes são transmitidos
muitas vezes por segundo por APs para indicar quais redes estão realizando \emph{broadcasting}. O modo monitor (\emph{monitor mode}) é um modo de operação em que um NIC consegue capturar todos os tipos de pacotes sem estar associado a um AP \cite{Acrylic} \cite{Wireshark2017b}. Dessa forma, é possível capturar todos os tipos, como os de \emph{probe request} que são enviados de dispositivos móveis para pontos de acesso para saber quais redes próximas estão disponíveis para se conectar.

Neste trabalho, um Raspberry Pi com um adaptador Wi-Fi habilitado no modo monitor captura pacotes \emph{probe request} de \meph{smartphones} para que ocorra a identificação de indivíduos (\autoref{smartphone-probe}).

\subsection{Kali Linux}
O sistema operacional Kali Linux \cite{kali} foi escolhido para o Raspberry Pi, pois possui ampla documentação
para uso em projetos de redes, além de ferramentas, como suporte a drivers de interfaces de rede que possam
ser habilitadas para o modo monitor, foi o caso do Ralink MT7601U.

\subsection{Tshark}
O protocolo Tshark é uma versão de terminal do protocolo
analisador de rede Wireshark \cite{Wireshark2017} \cite{Wireshark2017a}. Ele é
utilizado para analisar e filtrar (\emph{sniff}) e converter os dados dos
pacotes capturados pelo sensor em um arquivo. Esse protocolo foi
escolhido, pois permite realizar o estudo da rede a partir do recebimento de
pacotes e seus campos, além possuir ampla documentação, maturidade e exemplos
por ser uma tecnologia aberta.

\subsection{Raspberry Pi}
Para a detecção de dispositivos móveis um Raspberry Pi Model 3 B juntamente com um adaptador Wi-Fi são utilizados. O Raspberry foi escolhido,
pois oferece interface amigável de programação (Kali Linux); possui poder de processamento para receber os milhares de pacotes, pré-processá-los
e enviar para o servidor; possui entrada USB pode receber uma antena Wi-Fi e seu tamanho pequeno \cite{rpi2017}.

Outras opções foram consideradas por serem baratas, acessíveis e terem documentação aberta. Foi o caso do ESP8266 que possui um tamanho extremamente
reduzido e possui o custo médio de R\$15,00 \cite{Embarcados2015}, mas seu uso para este trabalho fica impossibilitado. Isso
ocorre, pois essa tecnologia não consegue ser habilitada para o modo monitor da interface de rede \cite{Puhl2016} \cite{Ferreira2016}.

Uma antena Wi-Fi (Ralink MT7601U) foi equipada no Raspberry para ampliar o alcance da captura já que o propósito do sistema é detecção em zonas que podem
apresentar esparcidade de indivíduos (espalhados) e já que ela pôde ser habilitada para o modo monitor (\autoref{modo-monitor}). A antena nativa
do Raspberry não conseguiu ser habilitada para o \emph{monitor mode}.

\subsection{Node.js}

\subsection{MongoDB e M-Lab}

\subsection{Visualização de dados}

\section{Métodos e Etapas}
Como a aplicação foi desenvolvida, etc. etc.. ordem de desenvolvimento

A implementação deste trabalho seguiu 3 etapas distintas: inicialmente, foi feito o levantamento teórico acerca de \emph{geomarketing} - devido à abrangência do tema, fez-se necessário definir qual seria o viés a ser seguido. Decidiu-se focar o desenvolvimento em conceitos relativos à contagem de pessoas e aferição de tráfego local. Foi realizado também um levantamento bibliográfico a respeito de empresas e projetos semelhantes ao proposto nesta monografia. 

A segunda etapa envolveu a escolha dos materiais e tecnologias a serem utilizados na concepção de um contator de pessoas. Foram realizados testes de compatibilidade entre hardware e Sistema operacional - no caso, utilizando-se como base o dispositivo portátil Raspberry Pi conforme \autoref{tecnologias-usadas}, fizeram-se testes para encontrar o SO adequado às necessidades do projeto, além de testes para escolha de uma antena Wi-Fi compatível com as especificações tecnológicas, uma vez que a antena nativa do Raspberry Pi não funciona em modo monitor, algo imprescindível na captura de pacotes específicos. Após selecionar hardware e SO, foram escolhidas as demais tecnologias e ferramentas a serem usadas no desenvolvimento da aplicação e visualização de dados, que se daria em forma de uma \emph{web page}: linguagens de programação, bancos de dados e servidores.

Na terceira etapa, durante e após o desenvolvimento e interligação de tecnologias e ferramentas, fizeram-se alguns testes em ambiente controlado para observar o comportamento do  contator e verificar se a captura de pacotes e contagem estavam ocorrendo como previsto. Foram realizados diversos ajustes a fim de otimizar a precisão da contagem, levando-se em conta o fato de que a aferição é uma estimativa média de tráfego e que existem limitações tecnológicas envolvidas. Após obtidos os resultados finais, fizeram-se testes extensivos (tanto em ambiente controlado quanto comercial) a fim de verificar-se a eficácia do dispositivo e também criar um banco de dados real, fornecendo os dados necessários ao \emph{website} para processamento e geração de informação.

\section{Dispositivo Móvel}
\label{smartphone-probe}
Para encontrar redes a que possa se conectar, um dispositivo móvel emite de tempos em tempos (depende da fabricante) pacotes do tipo \emph{probe request} (conceito explicado na seção \autoref{tecnologias-usadas}) para os APs próximos \cite{Meraki}. Todos os APs que receberem, responderão ao dispositivo (\emph{probe response} ou \emph{received}), então o aparelho descobrirá as redes ao redor disponíveis para conexão.




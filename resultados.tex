\chapter{Resultados e Discussão}
\label{resultados-discussao}

Este capítulo descreve os testes realizados em relação ao sistema e a discussão sobre cada um deles.

\section{Método de teste}
\label{metodo-teste}
Os testes realizados tiveram como principal objetivo verificar se o GSMART mede o tráfego
de pessoas numa determinada zona.

Os experimentos consistiram em deixar o sensor detectando dentro de zonas
selecionadas durante certos períodos de tempo. Buscou-se variar os ambientes de
detecção nos quesitos: concentração de pessoas e tempo de detecção. Para algumas
áreas menos concentradas foi possível registrar presencialmente o número de
pessoas por hora para comparação com os resultados do sistema. Em nenhum momento durante os experimentos
houve queda de energia, que impediria o escaneamento da zona.

Os métodos utilizados para aferir a precisão do sistema serão: comparação de médias e desvio padrão.

\section{Zona 1: LTIA}
\label{ltia-zone}

\begin{table}[h]
\centering
\caption{Contagem total na Zona LTIA}
\vspace{0.5cm}
\begin{tabular}{c|cc}
Dia & GSmart & Real \\
\hline                                          
24-07-2017 & x & x \\
25-07-2017 & x & x 
\end{tabular}
\end{table}

\section{Zona 2: HOMEJ}
\label{homej-zone}

Os testes desta zona foram realizados em um edifício residencial pequeno, com 3 andares e 6 apartamentos - por esse motivo, o alcance da antena capturou mais indivíduos do que os que se localizavam na área do apartamento (4 pessoas) durante as medições. 

Para exemplificação, são mostradas aqui as capturas coletadas numa sexta, entre 00:00 e 04:00 horas, executadas de hora em hora dentro desse intervalo, e no sábado, em 2 intervalos (também efetuando-se a captura de hora em hora): 12:00 às 17:00 horas e 20:00 às 22:00 horas.

\begin{table}[h]
\centering
\caption{Contagem total na Zona HOMEJ}
\vspace{0.5cm}
\begin{tabular}{c|cc}
Dia & GSmart & Real \\
\hline                                          
24-11-2017 & 12 & 4 \\
25-11-2017 & 49 & 4 
\end{tabular}
\end{table}


\section{Zona 3: HOMEC}
\label{homec-zone}

Blablabla

Bla Bla Bla bla bla bla bla bla

\begin{table}[h]
\centering
\caption{Contagem total na Zona HOMEC}
\vspace{0.5cm}
\begin{tabular}{c|cc}
Dia & GSmart & Real \\
\hline                                          
24-11-2017 & x & x \\
25-11-2017 & x & x 
\end{tabular}
\end{table}

\section{Zona 4: Camflam}
\label{camflam-zone}

\begin{table}[h]
\centering
\caption{Contagem total na Zona Camflam}
\vspace{0.5cm}
\begin{tabular}{c|cc}

Dia & GSmart & Real \\
\hline                                          
24-07-2017 & x & x \\
25-07-2017 & x & x 

\end{tabular}
\end{table}

\section{Avaliação de informações fornecidas}
\label{avalia-info}

\section{Avaliação de medição de tráfego}
\label{avalia-trafego}

%
% \section{Testes e validação do projeto}
%
% Para a validação do sistema proposto, executaram-se testes unitários, de integração e
% validação. Os principais objetivos dos testes foram:
%
%
% Os testes também tiveram objetivo de verificar se o volume de dados e/ou o tempo de captura interfeririam na eficiência do contador, já que foram realizadas verificações de tempo curtas (em torno de 5 a 10 minutos) e longas (3 horas ou mais ininterruptas).
%
% Os testes unitários e de integração do sistema resumem-se em:
%
%
%
% Já a validação focou-se em determinar se o sistema consegue ou
% não realizar a contagem e o tráfego de pessoas. Para tanto, foram feitos testes em
% ambientes controlados e não controlados: os controlados são zonas pequenas, em que se conhece previamente o número de pessoas, sendo possível conferir o resultado da detecção. Nos ambientes não-controlados, a medição do tráfego é feita sem que se saiba quantas pessoas poderão adentrar a zona de teste. A taxa de confiabilidade no sistema é baseada no desvio padrão dos testes realizados em ambiente controlado - o sistema final é considerado aplicável ou não conforme o desvio padrão determinado (dependendo se este encontra-se ou não dentro dos limites estabelecidos).
%
% O procedimento de testes ocorreu da seguinte maneira: inicialmente, foram feitos em ambiente controlado, a fim de observar o comportamento do sistema desenvolvido. Após todas as verificações controladas, testou-se o sistema em ambiente não-controlado, ou seja: um estabelecimento comercial - nessa etapa, o projeto verificou o desempenho do sistema em ambiente real, com grande quantidade de dispositivos móveis.
%
% Como ambiente controlado, foram escolhidos o LTIA (Laboratório de Tecnologia da Informação Aplicada), localizado na UNESP de Bauru/SP (contando-se com as devidas autorizações dos responsáveis) e também a residência das autoras deste trabalho. Percebeu-se que, nos testes realizados na zona do LTIA, a contagem ficou um pouco acima do esperado devido à grande quantidade de máquinas e dispositivos que se encontravam no recinto, consideravelmente superior à de pessoas. Nos testes realizados na moradia das autoras, o tráfego obtido mostrou-se bem mais próximo ao esperado, considerando-se houve a captura de pacotes originados em  apartamentos adjacentes, já que ambas residem em condomínio.
%
% Para verificação em ambiente não-controlado, realizaram-se testes no estabelecimento comercial Camflam Lanches, em Bauru/SP (também contando-se com as devidas autorizações). A captura de pacotes e geração de dados deu-se normalmente, e os resultados obtidos mostraram-se dentro do previsto. Neste ambiente, foram feitos somente testes longos, sem interrupção, com objetivo de aproximar-se ao máximo do que seria o ambiente real no qual o dispositivo de fato operaria.
%
% \section{Análise de Riscos}
% Alguns fatores e requisitos funcionais estão sujeitos a desvios de planos ou falhas técnicas, sendo necessário implementar um plano de contingência para garantir a acurácia do sistema, conforme descrito abaixo:
%
% \begin{itemize}
%   \item Falha da detecção de dispositivos (precisão): foram feitas duas formas de detecção através do protocolo Tshark - ambas garantem que os dados de uma e outra são verídicos - caso uma forma de detecção falhe, há outra para detectar os dispositivos móveis;
%   \item Processamento em servidor: o desenvolvimento do processamento de dados no servidor é complexo, considerando-se que trabalhamos com estatísticas. Optou-se então por uma \emph{cloud}, a fim de obter-se melhor desempenho e segurança possíveis;
%   \item Falhas de conexão: O Raspberry Pi está sujeito a perder a conexão com a rede e ficar impossibilitado de repassar os dados ao servidor - logo, um backup é armazenado dentro do aparelho: no caso de falha na conexão, quando a mesma retornar, os arquivos serão enviados ao servidor. Existe ainda a possibilidade do uso de um Modem 3G para garantir o envio.
% \end{itemize}

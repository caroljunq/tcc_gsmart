\chapter{Resultados e Discussão}
\label{resultados-discussao}

% terceira etapa, durante e após o desenvolvimento e interligação de
% tecnologias e ferramentas, fizeram-se alguns testes em ambiente controlado para
% observar o comportamento do  contator e verificar se a captura de pacotes e
% contagem estavam ocorrendo como previsto. Foram realizados diversos ajustes a
% fim de otimizar a precisão da contagem, levando-se em conta o fato de que a
% aferição é uma estimativa média de tráfego e que existem limitações tecnológicas
% envolvidas. Após obtidos os resultados finais, fizeram-se testes extensivos
% (tanto em ambiente controlado quanto comercial) a fim de verificar-se a eficácia
% do dispositivo e também criar um banco de dados real, fornecendo os dados necessários ao \emph{website} para processamento e geração de informação.

\section{Testes e validação do projeto}
Para a validação do sistema proposto serão realizados testes unitários, de integração e
validação. Os testes unitários e de integração do sistema resumem-se em:

\begin{itemize}
  \item detecção dos dispositivos móveis;
  \item comunicação entre Raspberry Pi e servidor;
  \item comunicação servidor e interface;
  \item processamento de dados capturados em dados desejados;
  \item determinar se sistema consegue contar pessoas.
\end{itemize}

Já os testes de validação são resumidos em determinar se o sistema consegue ou
não determinar a contagem e o tráfego de pessoas. Para tanto, serão feitos testes em
ambientes controlados e não controlados. Os controlados são
aquelas zonas em que sabe-se o número de pessoas, e então confere-se com o resultante da
detecção. Nos ambientes não-controlados, o tráfego de pessoas será testado.

A taxa de confiabilidade no sistema será baseada no desvio padrão dos testes
realizados em ambiente controlados.
O sistema final vai ser considerado aplicável ou não caso o desvio
padrão determinado fique dentro
dos limites estabelecidos.

Inicialmente, visa-se desenvolver os primeiros testes em ambiente
controlado, numa área pequena e com poucos dispositivos móveis, para verificar o
comportamento do sistema desenvolvido na medição do tráfego. Após testes
iniciais, pretende-se encontrar uma organização parceira que esteja dentro das
especificações necessárias e deseje conhecer melhor seu público alvo, cedendo
seu espaço e sua rede para alguns procedimentos e testes com a aplicação proposta
 - nessa etapa, o projeto busca verificar o desempenho do
sistema em ambiente real, com maior quantidade de dispositivos móveis.

\section{Análise de Riscos}
Considerando as premissas dos tópicos anteriores, há alguns itens e áreas que podem sofrer desvios ao longo do trabalho. Estes itens e seus planos
de contingência respectivamente são:

\begin{itemize}
  \item Falha da detecção de dispositivos (precisão): serão feitas duas formas de detecção através do protocolo Tshark, as duas garantem que os dados
  de uma e outra são verídicos, caso uma falhe há a outra para detectar os aparelhos móveis;
  \item Processamento no servidor é complexo: caso o desenvolvimento do processamento de dados no servidor seja muito complexo e considerando
  que trabalharemos com estatísticas, optar por uma \emph{cloud} seria uma opção;
  \item Raspberry Pi perder a conexão com a rede para mandar dados ao servidor: um backup dos dados será feito no aparelho, e então quando
  a conexão retornar, esses serão enviado ao servidor. Também há a possibilidade do uso de um Modem 3G para garantir o envio.
\end{itemize}

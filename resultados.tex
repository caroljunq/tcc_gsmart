\chapter{Resultados e Discussão}
\label{resultados-discussao}

Este capítulo descreve os testes realizados em relação ao sistema e a discussão sobre cada um deles.

\section{Método de teste}
\label{metodo-teste}
Os testes realizados tiveram como principal objetivo verificar a precisão e exatidão do GSMART em relação a
medição do tráfego de pessoas.

Os experimentos consistiram em deixar o sensor detectando dentro de zonas
selecionadas durante certos períodos de tempo (escolha aleatória e de acordo com a disponibilidade
dos donos). Buscou-se variar os ambientes de
detecção nos quesitos: concentração de pessoas e tempo de detecção. Para algumas
áreas menos concentradas foi possível registrar presencialmente o número de
pessoas por hora para comparação com os resultados do sistema. Em nenhum momento durante os experimentos
houve queda de energia, que impediria o escaneamento da zona.

Os métodos de avaliação estão centrados nos conceitos de: valor referência, média amostral e desvio padrão.
A variável quantitativa discreta identificada é o número de pessoas. O tratamento estatística adotado nos testes
é a inferência estatística (amostras) que é uma maneira rápida e econômica de fazer
inferência acerda da população \cite{Cabral2004}. As escolha do método de amostras
foi escolhido considerando que nem todos as horas foram monitoradas e as diferentes condições de
cada ambiente de teste.

\section{Terminologias estatísticas}
Antes de expor os testes e seus resultados é necessária uma revisão acerca de termos estatísticos segundo \citeonline{Cabral2004}. Eles são:
\begin{itemize}
    \item \textbf{valor verdadeiro}: o valor que obteríamos numa medição ideal, feita de condições
    perfeitas com instrumentos perfeitos e por operadores perfeitos;
    \item \textbf{exatidão:} a maior ou menor aproximação entre o resultado e o valor verdadeiro;
    \item \textbf{precisão:} está associada à dispersão dos valores resultantes da repetição das medições.
\end{itemize}

\section{Ambientes de teste}
Os ambientes de teste foram escolhidos de acordo com a disponibilidade de horário e autorização
para detecção. A condição que variou em cada um deles foi: concentração de pessoas e horários de detecção.
Outras características que influenciaram na detecção seram discutidos na \autoref{erros-influencia}.

\subsection{HOMEC}

\subsection{HOMEJ}
Os testes desta zona foram realizados em um edifício residencial pequeno, com 3
andares e 6 apartamentos - por esse motivo, o alcance da antena capturou mais
indivíduos do que os que se localizavam na área do apartamento (4 pessoas)
durante as medições.

Para exemplificação, são mostradas aqui as capturas coletadas numa sexta, entre
00:00 e 04:00 horas, executadas de hora em hora dentro desse intervalo, e no
sábado, em 2 intervalos (também efetuando-se a captura de hora em hora): 12:00
às 17:00 horas e 20:00 às 22:00 horas.

\subsection{LTIA}

\subsection{Camflam}

\section{Teste de precisão}

\section{Teste de exatidão}

\section{Erros e influenciadores na medição}
\label{erros-influencia}

% Durante as medições foram identificados e observados os tipos de erros: sistematico e grosseiro.
% identificar quais deles
% teoricos -> por simplificacao de modelos
%
%
% % --> grafico de dispersao mostra que curva é similar com o verdadeiro ou correlacao linear
% %
% % valor referencia para análise de exatidão --> media dos valores verdadeiros
% % quanto varia se varia,

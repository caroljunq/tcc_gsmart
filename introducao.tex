
\chapter{Introdução}
\label{introducao}

Desde os primeiros sistemas comerciais estabelecidos até a Revolução Industrial,
a maneira de se fazer negócios evoluiu drasticamente. Entretanto, nada se
compara à revolução gerada com o advento da computação: informatizar
organizações permitiu-nos coletar e armazenar volumes dados
gigantescos, gerando uma quantidade de informação humanamente impossível de
ser manipulada. Como consequência, foi imprescindível desenvolverem-se ferramentas e
métodos de mineração para que, dentre esse emaranhado de dados, fosse possível extrair-se informação relevante. 

Hoje, o conhecimento estatístico tornou-se elemento vital na tomada de decisões em negócios: a qualidade e abrangência das informações levantadas pode significar o sucesso ou fracasso na estratégia empresarial. Avaliar a expansão através de filiais, por exemplo, é um passo arriscado - se mal planejado, poderá incorrer em sérios prejuízos. Estudos detalhados acerca de geolocalização devem ser realizados numa região antes da instalação de uma nova loja. * incluir os 4Ps. 

Segundo \citeonline{Mangini2014}, a tomada de decisão em termos de
localização não pode ser feita de maneira aleatória e subjetiva, mas embasada em
um método ou ferramenta que permita determinar o melhor ponto ou o mais
adequado, de acordo com premissas objetivas e dentro de um arcabouço lógico,
considerando as possíveis variáveis que afetam aspectos relacionados ao usuário,
urbanismo e também relacionado à gestão e às políticas públicas. 

Neste contexto, o \emph{geomarketing} surge como grande tendência na gestão: reunindo conceitos em geografia espacial, estatística, gestão e marketing, tornou-se a mais abrangente ferramenta para visualização e análise do negócio como um todo - permite explorar vantagens locacionais, além de gerar um panorama completo de todas as camadas de uma instituição, identificando pontos fortes e fracos e auxiliando na definição das melhores estratégias de planejamento e decisão.

Neste trabalho, o foco do \emph{geomarketing} será voltado à exploração do tráfego
de pessoas através da criação do GSmart, uma ferramenta de contagem aliada às redes de
Internet sem fio e dispositivos móveis.

\section{Problema e justificativa}
O tráfego de pessoas já é largamente utilizado como técnica de \emph{geomarketing}, porém
no mercado predominam os softwares privados. Além disso, a maioria das ferramentas de contagem utiliza processamento de imagens, tecnologia de alto custo e demasiadamente sofisticada para aplicações mais simples. 

Visa-se, assim, criar uma solução em software livre, propiciando uma ferramenta gratuita e de fácil utilização voltada a profissionais da área de marketing, computação, administração, empresas e negócios, etc. Para tal, foram escolhidos como foco redes Wi-Fi e dispositivos móveis (celulares, tablets, etc) - isso se justifica por serem tecnologias acessíveis, relativamente baratas e amplamente utilizadas por qualquer pessoa hoje em dia. Além disso, aplicações de \emph{geomarketing} baseadas em redes sem fio são ainda pouco exploradas em pesquisas acadêmicas.

\section{Objetivos}
\label{objetivos}

\subsection{Objetivos Gerais}
Este trabalho tem como objetivo desenvolver um sistema que mede o tráfego de
pessoas em determinadas zonas através de rede Wi-Fi e dispositivos móveis.

\subsection{Objetivos específicos}
\begin{itemize}
  \item Estudar ferramentas de contagem de pessoas em ambientes e definir tecnologias para identificação e fornecimento de dados de usuários;
  \item Identificar a tecnologia responsável pela contagem de pessoas;
  \item Definir o modo como o número de indivíduos será agrupado para gerar o tráfego, indicar como os dados capturados serão agrupados e implementar interface para visualização dos dados gerados;
\end{itemize}

\section{Organização do trabalho}
O presente trabalho divide-se em capítulos, sendo este o primeiro (Introdução). Os próximos seguirão a ordem abaixo:

\begin{itemize}
  \item \textbf{Fundamentação Teórica:} apresentação dos conceitos teóricos envolvidos no trabalho, motivação de
  adoção de certas tecnologias para a construção do sistema e soluções semelhantes;
  \item \textbf{Materiais e Métodos:} ferramentas escolhidas para o desenvolvimento, métodos de testes, planos de contingência e módulos
  do sistema, arquitetura do sistema e construção;
  \item \textbf{Cronograma:} módulos que serão entregues nos respectivos períodos indicados.
\end{itemize}


\chapter{Introdução}
\label{introducao}

Desde os primeiros sistemas comerciais estabelecidos até a Revolução Industrial, a maneira de se fazer negócios evoluiu drasticamente - entretanto, nada se compara à revolução gerada com o advento da computação: a informatização das organizações não somente permtiu-nos coletar e armazenar volumes dados gigantescos, mas combiná-los e processá-los com agilidade e precisão antes nunca vistos. Com isso, geramos uma quantidade de informação humanamente impossível de ser manipulada, o que, por sua vez, demandou desenvolver-se ferramentas e métodos de filtragem e classificação dados, de modo a minerá-los e gerar informação relevante a uma organização, elemento hoje imprescindível no sucesso nas decisões estratégicas de qualquer empreendimento: avaliar se uma empresa deve pensar em expandir seu negócio (criar franquias, por exemplo) é um projeto arriscado - se mal planejado, pode levar uma organização a sofrer sérios prejuízos. Segundo \citeonline{Mangini2014}, a tomada de decisão em termos de localização não pode ser feita de maneira aleatória e subjetiva, mas embasada em um método ou ferramenta que permita
determinar o melhor ponto ou o mais adequado, de acordo com premissas objetivas e dentro
de um arcabouço lógico, considerando as possíveis variáveis que afetam aspectos
relacionados ao usuário, urbanismo e também relacionado à gestão e às políticas públicas. Neste contexto de união entre o marketing com noções de geografia e análise de vantagens
locacionais, o geomarketing surge como tendência para a determinação de pontos específicos
para a criação ou ampliação de uma empresa privada, mas que também pode ser aplicado em âmbito público. Em vista do previamente exposto, o presente trabalho optou por embasar-se em ferramentas e técnicas de análise de Geomarketing aliadas a redes de internet sem fios e dispositivos móveis (celulares e tablets)para captação e estudo de frequência de indivíduos em recintos específicos.

\section{Justificativa}
\label{justificativa}

Lorem ipsum

\section{Objetivos}
\label{objetivos}

\subsection{Objetivos Gerais}
Este trabalho tem como objetivo desenvolver um sistema que determina o tráfego de pessoas dentro de áreas através da conexão entre dispositivos móveis e redes
Wi-Fi, bem como identificar o perfil desses usuários quanto ao dispositivo que utilizam.

\subsection{Objetivos específicos}
\begin{itemize}
  \item Definir os motivos pelos quais uma organização utiliza o \emph{geomarketing};
  \item Identificar casos de uso do tráfego de indivíduos em ambientes como técnica
  de \emph{geomarketing};
  \item Estudar ferramentas de contagem de pessoas em ambientes;
  \item Definir as tecnologias para identificação e fornecimento de dados de usuários;
  \item Identificar a tecnologia responsável pela contagem de pessoas;
  \item Definir o modo como o número de indivíduos será agrupado para gerar o tráfego;
  \item Indicar como os dados capturados serão agrupados para gerar perfis de usuário;
  \item Implementar interface para apresentar o tráfego e os perfis das pessoas identificadas;
  \item Testar o sistema em ambientes controlados e não controlados;
  \item Realizar ajustes para garantir precisão do sistema desenvolvido.
\end{itemize}

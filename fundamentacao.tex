
\chapter{Fundamentação Teórica}
\label{fundamentacao-teorica}

\section {Geomarketing}
\label{Geom}
Segundo \citeonline{Sergio2005},\emph{Geomarketing} é o nome dado à área de gerenciamento de informação que incorpora as dimensões espaciais para auxilío à tomada de decisões dentro do domínio específico de mercado, o que permite levantar as características de uma determinada região e analisar seu potencial sócio-econômico. Pode ser entendido, assim, como uma ferramenta de análise estatística de dados, com intuito de localizar padrões que possam ser utilizados e combinados na elaboracao de indicadores, perfis de consumo e estratégias de negócios, de modo a gerar informação relevante na tomada de decisões. Geralmente, o servico é oferecido por consultorias especializadas - o  objetivo da empresa  contratante é a melhoria no desempenho de seu negócio.

O termo \emph{Geomarketing} é ainda pouco conhecido no Brasil, no entanto cada vez mais se populariza no âmbito dos negócios: segundo a revista \citeonline{Exame}, utilizado de forma amadora há 20 anos, o uso de ferramentas de localização geográfica evoluiu e alcançou importância dentro da estratégia de expansão das empresas: grupos como Coca-Cola e O Boticário usam o marketing geográfico e pequenas e médias empresas já começam a mirar em sistemas de busca com foco na geolocalização. Podemos citar como exemplo de pequeno negócio a empregar análise de Geomarketing um restaurante voltado à alimentação saudável em Natal/ RN - o objetivo foi verificar a distribuição geográfica de clientes e mapear áreas de influência para conhecer melhor a demanda do mercado. De acordo com \citeonline{Seabra2014}, esta investigação permitiu uma compreensão do fenômeno da área de influência e de variáveis que modelam seu comportamento. O estudo baseou-se em informações obtidas através dos softwares como \emph{Google Maps} para o georreferenciamento e análise dos dados - isso só foi possivel graças a fácil disponibilidade e barateamento da tecnologia atual: o \emph{Google Maps} é um exemplo de ferramenta de geolocalização bastante popular e acessível que, há alguns anos, não existia. 

Por outro lado, o acelerado desenvolvimento tecnológico e o crescimento de grandes centros urbanos criaram uma infinidade de possibilidades em aplicações para o \emph{Geomarketing}, tornando a ferramenta cada vez mais ampla e complexa. Um exemplo a ser citado nesse contexto é a aplicação do \emph{Geomarketing} como ferramenta de análise para criação de novas estações na CPTM (Companhia Paulista de Trens Metropolitanos). Segundo \citeonline{Mangini2014}, o modelo apresentou ser de grande valia por reduzir de forma substancial a subjetividade da escolha do local para uma nova estação e pode ainda ser utilizado como método para a definição de novas linhas férreas.

Podemos assim perceber a dimensão e importância do \emph{Geomarketing} hoje como referencial na tomada de decisões estratégias em todo tipo de organização, tornando-se aos gestores uma ferramenta valiosa,  a qual pode significar a diferença entre sucesso ou fracasso de um negócio. 

\section{Aplicação}
Diante do exposto na \autoref{Geom}, o presente trabalho visa utilizar técnicas de \emph{Geomarketing} e dispositivos tecnológicos na verificação e medição de frequência em áreas especificas, buscando analisar a demanda de acordo com a necessidade da organização, podendo-se avaliar a entrada de novos pontos estratégicos de atuação ou mesmo incrementar o alcance nos locais já existentes. Um exemplo de público-alvo poderia ser representado por \emph{shopping centers}, restaurantes, franquias, etc.  


\chapter{Fundamentação Teórica}
\label{fundamentacao-teorica}

\section {Geomarketing}
O Geomarketing pode ser entendido como uma ferramenta de análise estatística de dados, com intuito de localizar padrões que possam ser utilizados e combinados na elaboracao de indicadores, perfis de consumo e estrategias de negócios, de modo a auxiliar na tomada de decisões. Geralmente, o servico é oferecido por consultorias especializadas - o  objetivo da empresa que contratante é a melhoria no desempenho de seu negócio.

O Geomarketing é ainda pouco difundido no Brasil, no entanto cada vez mais se populariza no âmbito dos negócios: segundo \cite{Exame}, utilizado de forma amadora há 20 anos, o uso de ferramentas de localização geográfica evoluiu e alcançou importância dentro da estratégia de expansão das empresas: grupos como Coca-Cola e O Boticário usam o marketing geográfico e pequenas e médias empresas já começam a mirar em sistemas de busca com foco na geolocalização. 
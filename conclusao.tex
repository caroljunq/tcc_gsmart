\chapter{Conclusão}
\label{conclusao}

O foco principal deste trabalho foi disponibilizar um sistema \emph{opensource}
que aplique os conceitos de \emph{geomarketing} vistos, abrangendo o estudo de
redes de computadores (redes sem fio) e dispositivos móveis aplicados à
administração, com ênfase em gestão da informação. A partir conceitos de
\emph{geomarketing} provenientes de autores, consultorias e trabalhos
semelhantes, foi possível agrupar a contagem de pessoas por períodos e descobrir
informações relevantes para a tomada de decisão estratégica em negócios
(\autoref{importancia}).

Salientamos que, como a as soluções de \emph{geomarketing}
são, em maior parte, de âmbito privado, grande parte da fundamentação teórica
volta-se para consultorias, estudos de caso e trabalhos correlatos. Por essa
razão, não foi possível o levantamento de dados financeiros sobre custos do serviço. Em livros, muitos conceitos teóricos podem ser encontrados -
entretanto, não como são aplicados, existiram dificuldades no desenvolvimento do
sistema proposto, como decidir de que forma organizar a fase estatística na
agregação/ contagem de pessoas e processamento de dados: a precisão do sistema e
as informações coletadas devem ser relevantes e numerosas (generalização).

O sistema desenvolvido consegue realizar a contagem de pessoas em zonas
específicas, porém ressalvamos que, em áreas muito grandes, faz-se necessário
multiplicar o número de sensores de acordo com a cobertura necessitada, a fim de
manter-se resultados pontuais.

Outro fator é que, no experimento realizado no LTIA, a diferença entre o número
real de pessoas e o detectado foi grande:  observou-se que, em locais com alta
de densidade de máquinas e dispositivos fixos por pessoa, a eficácia da
estimativa é comprometida: como a quantia de dispositivos é muito mais alta do
que a de pessoas, esse fator pode inflar as estimativas, exibindo uma frequência
de pessoas acima do real. Logo, por se basear no \emph{MAC Address} de
dispositivos para contagem, o sistema não é indicado para medição em locais com
alta concentração de aparelhos fixos.

O contator se comporta bem em áreas onde existe alta frequência na entrada e saída de pessoas, como
restaurantes e lojas, em que os transeuntes geralmente portam um aparelho móvel
pessoal (\emph{smarthphone} ou \emph{tablet}). Portanto, o dispositivo não é indicado para espaços com poucas pessoas e muita quantidade de máquinas ao redor. O GSmart seria bastante útil na gestão de rede e dimensionamento de perfil tecnológico, por exemplo.

Foi observado também que, indivíduos inicialmente detectados no ambiente, ao irem embora, ainda permanecerão na contagem, ou seja: a saída de pessoas não é considerada, somente a chegada - se uma pessoa ficar por apenas 1 minuto no local, será contada pelo dispositivo. Para futura implementação, seria interessante capturar também o tempo de permanência de cada indivíduo/ MAC na área, além de utilizar esse período para validar se, por exemplo, uma pessoa que adentre um estabelecimento comercial de fato utilizou os serviços do local ou se está somente de passagem. Durante o tempo de detecção, ocorrem intervalos em que pessoas podem não ser detectadas, causados por interferências, obstáculos ou limitações tecnológicas, conforme descrito na \autoref{wifi} e \autoref{erros-influencia}.

Diante do exposto neste trabalho, apesar das limitações verificadas, concluímos
que o dispositivo cumpre o objetivo de medição de tráfego com baixa exatidão e precisão média,
além de prover
estatísticas comparativas e métricas. Além disso, o GSMART mostrou-se uma
ferramenta muito útil para estudantes, proporcionando recursos ao estudo de
\emph{geomarketing} à comunidade aberta e demais interessados.


\section{Trabalhos Futuros}
Como trabalhos futuros, propõe-se alguma melhorias para garantir maior exatidão e
precisão do sistema deste trabalho. Essas melhorias são:
\begin{itemize}
    \item \textbf{Tempo de estadia}: considerar um tempo mínimo de estadia dentro de uma zona para indivíduo ser
    contado;
    \item \textbf{Classificar dispositivos}: tentar diferenciar \emph{smartphones} de computadores comerciais;
    \item \textbf{Explorar comunicação GSM}: com autorização de empresas de telefonia, explorar comunicação GSM
    para não depender somente do Wi-Fi;
    \item \textbf{Detectar múltiplos aparelhos por pessoa}: através da potência de sinal e \emph{timestamp}
    de pacotes determinar se dispositivos tendem a pertencer a um mesmo indivíduo.
\end{itemize}

Além da melhoria no trabalho desenvolvido, indica-se o sistema construído como um potencial para gestão de redes
de computadores dentro de zonas urbanas e rurais onde é possível mapear grande quantidade de dispositivos com o
que já foi levantado neste trabalho. Além disso, pode-se aplicar o GSMART como ferramenta de decisões para mobilidade
urbana a partir do momento em que horas de pico de tráfego podem ser inferidas.

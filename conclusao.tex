\chapter{Conclusão}
\label{conclusao}

O foco principal deste trabalho foi disponibilizar um sistema \emph{opensource} que aplique os conceitos de \emph{geomarketing} vistos, abrangendo o estudo de redes de computadores (redes sem fio) e dispositivos móveis aplicados à administração, com ênfase em gestão da informação. A partir conceitos de \emph{geomarketing} provenientes de autores, consultorias e trabalhos semelhantes, foi possível agrupar a contagem de pessoas por períodos e descobrir informações relevantes para a tomada de decisão estratégica em negócios (\autoref{importancia}). 

O protótipo desenvolvido consegue realizar a contagem de pessoas em zonas específicas,
porém ressalvamos que, em áreas muito grandes, faz-se necessário multiplicar o número de sensores de acordo com a cobertura necessitada, a fim de manter-se resultados pontuais. Foi ainda observado que, em locais com alta de densidade de máquinas e dispositivos fixos por pessoa, como o LTIA (Laboratório de Tecnologia da Informação Aplicada), a eficácia da estimativa foi comprometida: como a quantia de dispositivos é muito mais alta do que a de pessoas, esse fator pode inflar as estimativas, exibindo uma frequência de pessoas acima do real. Logo, por se basear no \emph{MAC Address} de dispositivos para contagem, o sistema não é indicado para medição em locais com alta concentração de aparelhos fixos. O contator se comporta com eficácia em áreas onde existe alta mobilidade (grande entrada e saída de pessoas), como restaurantes e lojas, onde os transeuntes geralmente portam um dispositivo móvel pessoal (\emph{smarthphone} ou \emph{tablet}). 

Outro ponto a ser salientado é que, como a as soluções de \emph{geomarketing} são, em maior parte, de âmbito privado, grande parte da fundamentação teórica volta-se para consultorias, estudos de caso e trabalhos correlatos. Por essa razão, foi difícil também o levantamento de dados financeiros para estimarem-se os custos do serviço. Em livros, muitos conceitos teóricos podem ser encontrados - entretanto, não como são aplicados, existiram dificuldades no desenvolvimento do sistema proposto, como decidir de que forma organizar a fase estatística na agregação/ contagem de pessoas e processamento de dados: a precisão do sistema e as informações coletadas devem ser relevantes e numerosas (generalização).

\chapter{Conclusão}
\label{conclusao}

A partir conceitos de \emph{geomarketing} provenientes
de autores, consultorias e trabalhos semelhantes, é
possível agrupar a contagem de pessoas por períodos e descobrir informações relevantes e estratégicas
para a tomada de decisão de negócios (\autoref{importancia}). O foco principal deste trabalho
é disponibilizar um sistema
\emph{opensource} que aplique os conceitos de \emph{geomarketing} vistos, abrangendo o
estudo de redes de computadores (redes sem fio) e o uso de dispositivos móveis. O protótipo desenvolvido
consegue realizar a contagem de pessoas em zonas específicas,
mas é necessário multiplicar o número de sensores de acordo
com a área e cobertura, caso deseje-se dados mais pontuais. Outro ponto a ser salientado, é como a
as soluções de \emph{geomarketing} são, em maior parte, de âmbito privado, grande parte
da fundamentação teórica volta-se para consultorias, estudos de caso e trabalhos correlatos. Em livros, é
possível encontrar os conceitos, entretanto, não como são aplicados, dificultando o
desenvolvimento do sistema proposto. Essa dificuldade é refletida no modo como a fase
estatística será organizada.

Os principais desafios a serem enfrentados estão na área de agregação da
contagem de pessoas e processamento de dados, pois a precisão do sistema e as
informações coletadas terão que ser relevantes e numerosas (generalização).
Portanto, as dificuldades estão na fase estatística.
